% !TEX root = ../Main.tex

\chapter{\ChooseTranslation{Überblick Gesundheitssystem}{Overview Health System}}

% ==== Infos Menschlich gesammelt ====

% - Use case -- Anwenderbezogen (Gerda, 78 Jahre, hat einen Schlaganfall erlitten)
%   - Patient
%   - (Haus-)Arzt
%   - Krankenhaus/Klinikum
%   - Reha-Einrichtung
%   - Ambulanter Pflegedienst

% - Zwei Säulen der Krankenversorgung
%   - Niedergelassene Fachärzte in Praxen
%    * Ambulante langfristige Versorgung
%    * Lange Patient <-> Arzt-Bindung (Hausarzt)
%   - MVZ (Medizinische Versorgungszentren) (*neu*)
%   - Krankenhäuser
%    * Mehrtägige Therapie/Pflege unter Überwachung
%    * Komplexer Raum-, Geräte-, Personaleinsatz möglich
%    * Hochspezialisierte Medizin
%    * Fachübergreifende Behandlung durch räumliche Nähe

% 4-Felder:
%  - Bedarf
%   * Demographische Entwicklung
%   * Morbidität (Krankheitshäufigkeit)
%   * Präferenz (Patientenwunsch)
%  - Umsetzung
%   * Leistungserbringer
%   * Leistungsträger
%   * Innovation
%  - Rahmenbedingungen
%   * Gesetzgebung
%   * Ressourcen (Personal, Finanzen, Technik)
%   * Struktur (Infrastruktur, Organisation)
%  - Outcome
%   * Lebenserwartung
%   * Lebensqualität
%   * Ressourcenverbrauch

% ``Follow the Money''
%   **Wer** zahlt **wem** **wieviel** für **was**?
%   **Wer** entscheidet das?
%   Was taugt das?

% Sozialversicherung
%   - Krankenversicherung
%     * §5 SGB V & §2 KVLG
%     * seit 1989 in aktueller Form
%     * Krankenversicherung als Solidargemeinschaft
%     * Gesundheit der Versicherten zu erhalten, wiederherzustellen oder ihren Gesundheitszustand zu bessern
%   - Pflegeversicherung
%     * §20 f. SGB XI
%   - Rentenversicherung
%     * §1 SGB VI & §1 ALG
%   - Arbeitslosenversicherung
%     * §25 ff. SGB III
%   - Unfallversicherung
%     * §2 SGB VII
%
% Frage: Wer ist Krankenversicherungspflichtig?
%   - Arbeitnehmer
%   - Rentner
%   - Arbeitslose
%   - Studenten
%   - Selbständige
%   - ...
% -> Fast jeder, der in Deutschland lebt

% Grundprinzipien gesetzliche Krankenversicherung
%   - Solidaritätsprinzip (Gesunde zahlen für Kranke)
%   - Sachleistungsprinzip (keine Barauszahlung)
%   - Selbstverwaltungsprinzip (Krankenkassen)
%   - Risikostrukturausgleich (Risikogerechte Verteilung der Gelder)
%   - Wirtschaftlichkeitsgebot (Sparsamkeit)
%   - Subsidiarität (Hilfe zur Selbsthilfe)
%
% Solidaritätsprinzip
%   - Beitragshöhe abhängig vom Einkommen
%   - Arbeitgeber und Arbeitnehmer zahlen je zur Hälfte
%   -> Ziel: Risikoausgleich zwischen:
%     * Jung und Alt
%     * Gesund und Krank
%     * Reich und Arm
%
% Sachleistungsprinzip
%   - Keine Barauszahlung
%   - Direkt an Leistungserbringer von KK gezahlt
%   ! Ausnahme: Privatversicherte (erstatten Rechnungen)
%
% Risikostrukturausgleich
%   - Krankenkassen mit höheren Ausgaben erhalten mehr Finanzmittel
%     * z.B. durch Diskrepanzen in Alter, Geschlecht, Morbidität
%
%
% Beitragsfinanzierte <-> Steuerfinanzierte Systeme
%   - Beitragsfinanziert:
%     * Deutschland, Schweiz, Niederlande
%     * Beiträge abhängig vom Einkommen
%     * Arbeitgeber und Arbeitnehmer zahlen
%   - Steuerfinanziert:
%     * Großbritannien, Schweden, Kanada
%     * Steuern finanzieren Gesundheitssystem
%     * Keine direkten Beiträge
%     * Gesundheitssystem als Staatsaufgabe
%
%
% GKV <-> PKV
%   - GKV (Gesetzliche Krankenversicherung)
%     * Bei < 4425€/Monat pflichtversichert
%     * Bei > 4425€/Monat freiwillig versichert
%     * Beitrag abh. vom Einkommen
%     * Sachleistungsprinzip
%     * Solidaritätsprinzip
%   - PKV (Private Krankenversicherung)
%     * Bei > 4425€/Monat möglich
%     * Beitrag abh. vom Gesundheitsrisiko
%     * Beiträge steigen somit bspw. im Alter
%     * Bessere Leistungen möglich
%     * Kostenerstattung
%     * Äquivalenzprinzip (Beitrag = Leistung)
%
% Leistungen der GKV:
% - Normale Kontrolluntersuchungen und Standardimpfungen
% - Haus- oder Facharzt
% - Krankenhaus
% - Medikamente
% - Krebsvorsorge-Untersuchungen
% - Heilmittel
% - Hilfsmittel
% - Kinderkrankengeld
% - Zahnbehandlungen
% - Zahnersatz
% - Kieferorthopädie
% - Fahrtkosten
% - Ambulante Kur / Reha
% - Stationäre Kur / Reha
% - Psychotherapie
% - Alternative Behandlungsmethoden
% - Schutz im Ausland
% - Beitragsrückerstattung bei Leistungsfreiheit
% - Kostenerstattung
% - Patientenquittungen
% - Chronische Krankheiten/DMP
%
% Aufgaben des BMG (Bundesministerium für Gesundheit)
%   - Erarbeitung von Gesetzesentwürfen, Rechtsverordnungen und Verwaltungsvorschriften
%   - Die Leistungsfähigkeit der Gesetzlichen Krankenversicherung sowie der Pflegeversicherung zu erhalten, zu sichern und fortzuentwickeln.
%   - Reform des Gesundheitssystems
%   - Qualität des Gesundheitssystems weiterzuentwickeln
%   - Interessen der Patientinnen und Patienten zu stärken
%   - die Wirtschaftlichkeit zu gewährleisten
%   - die Beitragssätze zu stabilisieren
% 
% ==== Ende Infos aus menschlicher Quelle ====

% ==== Von ChatGPT generiert ====
% Definition von Gesundheit
% - Gesundheit als Idealzustand mit völligem Wohlbefinden ohne körperliche, psychische und soziale Störungen
% - Gesundheit als persönliche Stärke, die auf körperlichen und psychischen Eigenschaften beruht
% - Gesundheit als Leistungsfähigkeit zur Erfüllung gesellschaftlicher Anforderungen
% - Gesundheit als Gebrauchsgut (Ware), das hergestellt und "eingekauft" werden kann

% Weitere Faktoren der Gesundheit
% - Prävention
% - Infektionsschutz
% - Bildung
% - Selbstversorgung
% - Soziale Netzwerke
% - Apps
% - Hygiene
% - Arbeit
% - Frieden
% - Wohlstand

% Use Case Gesundheitssystem (Gerda M., 78 Jahre)
% - Symptome: Schwindel, Vergesslichkeit, Sprachstörungen
% - Diagnosen: Durchblutungsstörung des Gehirns, Vaskuläre Demenz, Bluthochdruck, Halbseitige Lähmung mit Neglect
% - Pflegegrad 3

% Zwei Säulen der Krankenversorgung
% I) Niedergelassene Fachärzte in Praxen:
% - ambulante langfristige Versorgung
% - lange Patient-Arztbindung (Hausarzt)
% - Krankengeschichte bekannt
% - effiziente individuelle Betreuung möglich
% II) Krankenhäuser:
% - mehrtägige Therapie/Pflege unter Überwachung
% - komplexer Raum-, Geräte-, Personaleinsatz möglich
% - Hochspezialisierte Medizin
% - fachübergreifende Behandlung durch räumliche Nähe

% Grundprinzipien der gesetzlichen Krankenversicherung (GKV)
% - Solidaritätsprinzip: Beitragshöhe richtet sich nach dem Einkommen (14,6% des Einkommens + individueller Zusatzanteil je Krankenkasse)
% - Arbeitgeber übernimmt die Hälfte der Kosten
% - Sachleistungsprinzip: GKV-Versicherte erhalten Leistungen grundsätzlich bargeldlos durch Vorlage der Krankenkassenkarte
% - Risikostrukturausgleich: Krankenkassen mit höheren Ausgaben erhalten mehr Finanzmittel (z.B. nach Alter, Geschlecht, Morbidität)

% Vergleich GKV und PKV
% - GKV: Pflichtversicherung bei Einkommen < 4425€/Monat, solidarische Beiträge, Sachleistungsprinzip, ca. 11% privatversichert
% - PKV: freiwillige Versicherung bei Einkommen > 4425€/Monat, Beiträge abhängig vom Gesundheitsrisiko, Kostenerstattung

% Organe der Selbstverwaltung im Gesundheitswesen
% - Bundesgesundheitsministerium (BMG)
% - Gemeinsamer Bundesausschuss (GBA)
% - Deutsche Krankenhausgesellschaft (DKG)
% - Kassenärztliche Bundesvereinigung (KBV)
% - Gesetzliche Krankenversicherungen (GKV)
% - Private Krankenversicherungen (PKV)
% - Bundesärztekammer (BÄK)
% - Landesärztekammern
% - Hartmannbund

% Herausforderungen im Gesundheitssystem
% - Ziel: Volksgesundheit als Systemziel
% - hohes Behandlungsniveau für alle ist teuer
% - neueste Medizintechnik und Medikamente
% - wirksamste Therapien
% - 24/7 Verfügbarkeit
% - hoher Ausbildungs- und Qualifikationsstand
% - langfristige Bezahlbarkeit von Leistungen

% Take home questions
% - Wie kann man Gesundheit definieren?
% - Wie unterscheiden sich gesetzliche und private Krankenkassen?
% - Welche Grundprinzipien hat das Gesundheitssystem?
% - Welche Organe der Selbstverwaltung kennen Sie?
% - Wie wird im ambulanten vs. im stationären Bereich abgerechnet?
% - Welche Abschnitte im Kompetenzerwerb von Medizinerinnen und Medizinern gibt es und wer ist für sie zuständig?

% ==== Ende ChatGPT ====