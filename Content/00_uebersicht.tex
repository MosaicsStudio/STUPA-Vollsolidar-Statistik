% !TEX root = ../main.tex

\newcommand{\germanQuote}[1]{„\textit{#1}“}

%----------------------------------------
%   SECTION TITLE
%----------------------------------------

\label{chap:introduction}

%----------------------------------------
%   SECTION CONTENT
%----------------------------------------

\chapter{Rahmen}
\label{sect:frame}
% Hier wird die Situation/Problematik beschrieben, zu deren Verständnis bzw. Lösung die empirische Erfassung der Gegebenheiten erforderlich ist: Eben eine Studie. Dabei wird auch die Studienart Längs‐/bzw. Querschnitt beschrieben. Da in der Aufgabenstellung eine Längsschnittstudie gefordert ist, müssen beide Aspekte dargelegt werden.
Ziel ist es eine Querschnittstudie durchzuführen, um die Meinungen der Studierenden zum Deutschlandticket im Vollsolidarmodell authentisch zu erfassen. (Statt Naldo im Teilsolidarmodell)

Unser Ziel dabei ist es, einen umfassenden Überblick über Zustimmung, Ablehnung, Skepsis oder Begeisterung zu gewinnen. Diese Studie soll uns helfen, fundiert zu entscheiden, ob und wie das Deutschlandticket unseren Alltag beeinflussen soll. Durch die direkte Erfassung der aktuellen Stimmungslage der Studierendenschaft wollen wir eine solide Basis für eine informierte Entscheidung schaffen.

\chapter{Fragestellung}
\label{sect:question}
% In zwei bis drei Sätzen wird dargelegt welche konkrete Frage mit Hilfe der Studie beantwortet werden soll. Hier ist die Beschränkung auf maximal drei Sätze absolut bindend. Wenn die Fragestellung zu viel Text erfordert ist Sie nicht konkret genug. Dann ist es wahrscheinlich, dass bereits der Fragebogen zur Datenerhebung zu unspezifisch und damit unbrauchbar wird.
Wie hoch ist die Akzeptanz unter den Studierenden der Hochschule Reutlingen für ein Deutschlandticket im Vollsolidarmodell?

\chapter{Vorgehensweise und Methode}
\label{sect:methods}
% Hier werden die geplanten Maßnahmen beschrieben mit denen die Fragestellung beantwortet werden soll, z.B. Art und Weise der Durchführung, wie wird der Längsschnitt realisiert etc.
Wir werden ein (selbst aufgesetztes) Online Tool verwenden. Ziel ist es diese Umfrage über die Email des STUPAs und die neue App der Hochschule \germanQuote{MyHSRT} an die Studenten zu verteilen. Zusätzlich soll es Plakate auf dem Campus geben. Eine weitere  Maßnahme ist die Beteiligung des Hochschulmarketings und der Mobilitätsbeauftragten Frau  Sabine Merkens. 
Dies gewährleistet eine Beteiligung aus allen Fakultäten der Hochschule, und somit die spätere Verwendbarkeit für das Studierendenparlament.

\chapter{Stichprobe und Datenerhebung}
\label{sect:data}
% Wichtigster methodischer Aspekt in diesem Teil ist die Struktur der Stichprobe. Hier wird dargelegt welche Personen/Entitäten aus der Grundgesamtheit mit welchen Eigenschaften jeweils eine Beobachtung liefern. Die Zusammensetzung der Stichprobe muss plausibel auf die Fragestellung abgestimmt sein. Dabei sind insbesondere Annahmen über Verfügbarkeit im Hinblick auf die Durchführung zu machen.

Allgemeine Kenntnis und Meinung:

\begin{displayquote}
\germanQuote{Was ist dein Wissensstand zum Deutschlandticket und dem Vollsolidarmodell?}
\end{displayquote}

\begin{displayquote}
\germanQuote{Wie stehst du zur Einführung des Deutschlandtickets an unserer Hochschule im Vollsolidarmodell?} (Von 1 \germanQuote{sehr dagegen} bis 5 \germanQuote{sehr dafür})
\end{displayquote}
Nutzung öffentlicher Verkehrsmittel:
\begin{displayquote}
\germanQuote{Wie häufig nutzt du aktuell öffentliche Verkehrsmittel für den Weg zur Hochschule oder in deiner Freizeit?} (Von \germanQuote{Täglich} bis \germanQuote{Nie})
\end{displayquote}
Deutschlandticket:
\begin{displayquote}
\germanQuote{Hältst du den festgelegten Beitrag für das Deutschlandticket im Vollsolidarmodell für angemessen?} (\germanQuote{Ja} oder \germanQuote{Nein})
\end{displayquote}

\begin{displayquote}
\germanQuote{Wenn nein, welcher Betrag wäre deiner Meinung nach angemessen?} (\textit{Numerisch})
\end{displayquote}

Erwartete Auswirkungen:

\begin{displayquote}
\germanQuote{Erwartest du, dass die Einführung des Deutschlandtickets im Vollsolidarmodell deine Nutzung öffentlicher Verkehrsmittel verändern würde?} (Von \germanQuote{Nein, gar nicht} bis \germanQuote{Ja, sehr})
\end{displayquote}

\begin{displayquote}
\germanQuote{Wie bewertest du das Prinzip, dass alle Studierenden zum Deutschlandticket beitragen, auch wenn sie es möglicherweise nicht nutzen?} (Von \germanQuote{Sehr Positiv} bis \germanQuote {Sehr Negativ})
\end{displayquote}

\chapter{Mögliche Ergebnisse}
\label{sect:results}
% Hier wird dargestellt welche möglichen Ergebnisse im Hinblick auf die Fragestellung die Studie liefern kann. Wichtig! Es geht nicht um das erwartete Ergebnis sondern um die möglichen theoretischen Erkenntnisse die gewonnen werden können; sowohl im positiven als auch negativen Sinne.
\bigskip

Folgende Ergebnisse sind möglich:
Die Studierendenschaft möchte...
\begin{enumerate}
    \item das Deutschlandticket im Vollsolidarmodell für alle Studierenden.
    \item beim bisherigen Naldo Ticket im Teilsolidarmodell bleiben.
    \item lehnt ein Solidarmodell jeglicher Art ab.
\end{enumerate}
